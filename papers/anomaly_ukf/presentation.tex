\documentclass{beamer}
\usetheme{Madrid}
%\usecolortheme{crane}

% Temple University Color Scheme
\definecolor{TempleCherry}{RGB}{164,30,53}
\definecolor{TempleWhite}{RGB}{255,255,255}
\definecolor{TempleGray}{RGB}{100,100,100}

\setbeamercolor{structure}{fg=TempleCherry}
\setbeamercolor{palette primary}{bg=TempleCherry,fg=TempleWhite}
\setbeamercolor{palette secondary}{bg=TempleCherry!80,fg=TempleWhite}
\setbeamercolor{palette tertiary}{bg=TempleCherry!60,fg=black}
\setbeamercolor{titlelike}{parent=palette primary}
\setbeamercolor{frametitle}{bg=TempleCherry,fg=TempleWhite}
\setbeamercolor{block title}{bg=TempleCherry,fg=TempleWhite}
\setbeamercolor{block body}{bg=TempleCherry!10,fg=black}

% Packages
\usepackage{amsmath}
\usepackage{amssymb}
\usepackage{graphicx}
\usepackage{booktabs}
\usepackage{subcaption}
\usepackage{tikz}
\usetikzlibrary{positioning,shapes.geometric,arrows.meta}

% Graphics path
\graphicspath{{figures/}}

% Bibliography
\usepackage[style=apa,natbib=true]{biblatex}
\addbibresource{references.bib}

% Presentation metadata
\title[MEMS-GeoNav: UKF]{Navigation in GNSS-Denied Environments Using MEMS-Grade Sensors and Geophysical Anomalies}
\subtitle{A UKF Approach}
\author{James Brodovsky \and Philip Dames}
\institute{Temple University}
\date{January 6, 2026}

% Logo placeholder - uncomment when available
 \titlegraphic{\includegraphics[width=2cm]{example-image-a}}

\begin{document}

% ============================================================
% TITLE SLIDE
% ============================================================
\begin{frame}
\titlepage
\end{frame}

% ============================================================
% MOTIVATION: GNSS VULNERABILITIES
% ============================================================
\begin{frame}{The GNSS Dependency Problem}
\begin{columns}
\begin{column}{0.5\textwidth}
\textbf{Modern navigation is critically dependent on GNSS}
\pause
\vspace{0.5cm}

\textbf{Vulnerabilities:}
\begin{itemize}
    \item Urban canyons \& multipath
    \item Indoor/underground environments
    \item Underwater operations
    \item Electronic warfare (jamming/spoofing)
\end{itemize}
\end{column}
\begin{column}{0.5\textwidth}
\centering
% Placeholder for GNSS vulnerability illustration
\fbox{\parbox{0.9\textwidth}{\centering\vspace{2cm}\textit{[GNSS Vulnerability\\Illustration]}\vspace{2cm}}}
\end{column}
\end{columns}
\end{frame}

% ============================================================
% MOTIVATION: COST GAP
% ============================================================
\begin{frame}{The Cost-Performance Gap}
\begin{block}{Historical Geophysical Navigation}
High-grade systems only:
\begin{itemize}
    \item Navigation/Tactical-grade IMUs (\$10K - \$100K+)
    \item Dedicated gravimeters and magnetometers
    \item Military \& aerospace applications
\end{itemize}
\end{block}
\pause

\begin{block}{The Opportunity}
Consumer platforms need GNSS-independent navigation:
\begin{itemize}
    \item Smartphones, drones, small autonomous vehicles
    \item MEMS-grade IMUs (\$10 - \$1K)
    \item Low SWaP constraints
\end{itemize}
\end{block}
\pause

\vspace{0.3cm}
\centering
\textbf{\color{TempleCherry}Can geophysical navigation work with low-cost sensors?}
\end{frame}

% ============================================================
% DATASET
% ============================================================
\begin{frame}{MEMS-Nav Dataset}
\begin{columns}
\begin{column}{0.55\textwidth}
\textbf{Data Collection:}
\begin{itemize}
    \item Smartphone-grade MEMS sensors
    \item Sensor Logger application
    \item Long-duration highway trajectories
    \item 30 min - several hours each
    \item Tens to hundreds of kilometers
\end{itemize}
\pause

\vspace{0.3cm}
\textbf{Sensors:}
\begin{itemize}
    \item 3-axis accelerometer
    \item 3-axis gyroscope
    \item 3-axis magnetometer
    \item Barometer
    \item GNSS receiver
\end{itemize}
\end{column}
\begin{column}{0.45\textwidth}
\centering
% Placeholder for trajectory map
\fbox{\parbox{0.9\textwidth}{\centering\vspace{2.5cm}\textit{[Sample Trajectory\\Map]}\vspace{2.5cm}}}

\vspace{0.3cm}
\small
\textit{13 trajectories evaluated}
\end{column}
\end{columns}
\end{frame}

% ============================================================
% SYSTEM ARCHITECTURE
% ============================================================
\begin{frame}{System Architecture}
\centering
\includegraphics[width=0.95\textwidth]{system_architecture.png}

\vspace{0.5cm}
\begin{itemize}
    \item \textbf{Standard INS:} 15-state UKF (9 navigation + 6 IMU bias states)
    \item \textbf{Geophysical-Aided:} +1 anomaly bias state per measurement
\end{itemize}
\end{frame}

% ============================================================
% GNSS DEGRADATION SCENARIO
% ============================================================
\begin{frame}{GNSS Degradation Scenario}
\begin{block}{Simulated GNSS Interference}
Realistic model of low-quality GNSS (urban canyons, multipath, jamming):
\end{block}

\begin{columns}
\begin{column}{0.5\textwidth}
\textbf{AR(1) Correlated Errors:}
\begin{itemize}
    \item Position: $\rho = 0.99$, $\sigma = 5$ m
    \item Velocity: $\rho = 0.95$, $\sigma = 5$ m/s
    \item Slowly drifting biases
\end{itemize}
\pause

\vspace{0.3cm}
\textbf{Covariance Inflation:}
\begin{itemize}
    \item $15\times$ advertised accuracy
    \item Simulates poor DOP reporting
\end{itemize}
\end{column}
\begin{column}{0.5\textwidth}
\centering
% Placeholder for degradation visualization
\fbox{\parbox{0.9\textwidth}{\centering\vspace{2cm}\textit{[AR(1) Error\\Visualization]}\vspace{2cm}}}
\end{column}
\end{columns}
\end{frame}

% ============================================================
% GEOPHYSICAL AIDING
% ============================================================
\begin{frame}{Geophysical Measurement Integration}
\begin{columns}
\begin{column}{0.5\textwidth}
\textbf{Gravity Anomaly Aiding:}
\begin{itemize}
    \item IGPP Earth Free-air Anomaly Map
    \item 1 arc-minute resolution ($\sim$1.8 km)
    \item Mean residual: 5.97 mGal
    \item Std: 48.93 mGal
\end{itemize}
\pause

\vspace{0.5cm}
\textbf{Magnetic Anomaly Aiding:}
\begin{itemize}
    \item World Digital Magnetic Anomaly Map (WDMAM)
    \item 3 arc-minute resolution ($\sim$5.6 km)
    \item Mean residual: 40,364 nT
    \item Std: 41,303 nT
\end{itemize}
\end{column}
\begin{column}{0.5\textwidth}
\centering
\includegraphics[width=\textwidth]{anomaly_differences.png}

\small
\textit{Measurement residual distributions}
\end{column}
\end{columns}
\end{frame}

% ============================================================
% RESULTS: BEST GRAVITY PERFORMER
% ============================================================
\begin{frame}{Results: Gravity Aiding Success}
\centering
\textbf{Trajectory 2023-08-04\_214758}

\includegraphics[width=0.8\textwidth]{gravity/2023-08-04_214758_rmse.png}

\vspace{0.3cm}
\textbf{\color{TempleCherry}RMSE Improvement: -1,170 meters}
\end{frame}

% ============================================================
% RESULTS: BEST MAGNETIC PERFORMER
% ============================================================
\begin{frame}{Results: Magnetic Aiding Success}
\centering
\textbf{Trajectory 2025-06-27\_11-54-35}

\includegraphics[width=0.8\textwidth]{magnetic/2025-06-27_11-54-35_rmse.png}

\vspace{0.3cm}
\textbf{\color{TempleCherry}RMSE Improvement: -550 meters}
\end{frame}

% ============================================================
% RESULTS: DUAL-MODAL SUCCESS
% ============================================================
\begin{frame}{Results: Dual-Modal Performance}
\centering
\textbf{Trajectory 2025-07-31\_23-36-03}

\begin{columns}
\begin{column}{0.5\textwidth}
\centering
\includegraphics[width=\textwidth]{gravity/2025-07-31_23-36-03_rmse.png}
\small
Gravity: -688m RMSE
\end{column}
\begin{column}{0.5\textwidth}
\centering
\includegraphics[width=\textwidth]{magnetic/2025-07-31_23-36-03_rmse.png}
\small
Magnetic: -632m RMSE
\end{column}
\end{columns}

\vspace{0.3cm}
\textbf{\color{TempleCherry}Both modalities provide substantial improvements}
\end{frame}

% ============================================================
% RESULTS: FAILURE CASE ANALYSIS
% ============================================================
\begin{frame}{Results: Failure Mode Analysis}
\centering
\textbf{Trajectory 2023-08-09\_163741 (Outlier)}

\begin{columns}
\begin{column}{0.5\textwidth}
\centering
\includegraphics[width=\textwidth]{gravity/2023-08-09_163741_rmse.png}
\small
Gravity: +8,659m error
\end{column}
\begin{column}{0.5\textwidth}
\centering
\includegraphics[width=\textwidth]{magnetic/2023-08-09_163741_rmse.png}
\small
Magnetic: +8,635m error
\end{column}
\end{columns}

\vspace{0.2cm}
\begin{itemize}
    \item Catastrophic initial divergence
    \item Indicates need for integrity monitoring and robust initialization
\end{itemize}
\end{frame}

% ============================================================
% SUMMARY STATISTICS
% ============================================================
\begin{frame}{Summary Statistics Across All Trajectories}
\centering
\includegraphics[width=0.9\textwidth]{summary_statistics.png}

\vspace{0.3cm}
\begin{itemize}
    \item \textbf{Median Improvement:} Gravity -3.47m, Magnetic -10.34m
    \item \textbf{6-8 trajectories} showed improvement with gravity aiding
    \item \textbf{4-11 trajectories} showed improvement with magnetic aiding (metric-dependent)
\end{itemize}
\end{frame}

% ============================================================
% KEY TAKEAWAYS
% ============================================================
\begin{frame}{Key Findings}
\begin{block}{Primary Achievement}
\textbf{First demonstration} of geophysical navigation feasibility using only:
\begin{itemize}
    \item MEMS-grade sensors (\$10-\$1K)
    \item Open-source geophysical maps
    \item Smartphone-class hardware
\end{itemize}
\end{block}
\pause

\begin{block}{Performance Highlights}
\begin{itemize}
    \item Error reductions: tens to hundreds of meters on favorable trajectories
    \item Best case: >1,100m RMSE improvement
    \item Median improvements: 3-10 meters across all trajectories
    \item Despite MEMS noise and limited map resolution
\end{itemize}
\end{block}
\pause

\begin{block}{Significance}
Opens pathway from niche military/aerospace applications to \textbf{mass-market consumer and commercial platforms}
\end{block}
\end{frame}

% ============================================================
% FUTURE WORK
% ============================================================
\begin{frame}{Future Directions}
\begin{enumerate}
    \item \textbf{Integrity Monitoring}
    \begin{itemize}
        \item Detect and mitigate failure modes
        \item Adaptive measurement weighting
        \item Innovation-based outlier rejection
    \end{itemize}
    \pause
    
    \item \textbf{Observability Analysis}
    \begin{itemize}
        \item Characterize which regions/trajectories benefit most
        \item Map-matching performance metrics
        \item Geophysical gradient requirements
    \end{itemize}
    \pause
    
    \item \textbf{Multi-Modal Fusion}
    \begin{itemize}
        \item Combine gravity + magnetic measurements
        \item Terrain-based aiding
        \item Enhanced map resolutions
    \end{itemize}
    \pause
    
    \item \textbf{Real-Time Implementation}
    \begin{itemize}
        \item On-device processing
        \item Computational efficiency
        \item Battery/power constraints
    \end{itemize}
\end{enumerate}
\end{frame}

% ============================================================
% REFERENCES
% ============================================================
\begin{frame}[allowframebreaks]{References}
\printbibliography
\end{frame}

% ============================================================
% ACKNOWLEDGMENTS
% ============================================================
\begin{frame}{Acknowledgments}
\begin{itemize}
    \item MEMS-Nav Dataset: \url{https://doi.org/10.5281/zenodo.17582434}
    \item Strapdown-rs Software: \url{https://github.com/jbrodovsky/strapdown-rs}
    \item IGPP Gravity Anomaly Maps (EGM2008)
    \item World Digital Magnetic Anomaly Map (WDMAM)
    \item Generic Mapping Tools (GMT)
\end{itemize}

\vspace{0.5cm}
\centering
\textbf{Questions?}

\vspace{0.3cm}
\small
James Brodovsky: james.brodovsky@temple.edu
\end{frame}

\end{document}
